%% ESTE TEMPLATE FOI ADAPTADO A PARTIR DA CLASSE ppgccufscar
%% criada pelo Programa de Pós Graduação em Ciência da Computação da UFSCAR
%%CRÉDITOS:Profa. Sandra Fabbri.
%% A classe ppgccufscar foi feita por Daniel Beck,
%% Daniel Bruno e Prof. Marcio

%%
%% Opcoes que podem ser passadas 'a classe
%%
%% Todas as opcoes da classe abnt (AbnTeX) sao validas.
%% Outras opcoes sao: quali e tese (dissertacao e' padrao)
%%

\documentclass[quali]{mpit}
%\renewcommand*\sectionmark[1]{\markboth{#1}{}}
%\renewcommand*\subsectionmark[1]{\markright{#1}}

%% pacotes que deseja usar
%% pacotes incompativeis sao:
%%   qualquer pacote de citacoes, como natbib, apalike, cite, etc.
%% o pacote babel ja vem carregado com ingles e portugues
\usepackage{verbatim}
\usepackage[utf8]{inputenc}
\usepackage{graphicx}

%% Fix numeração das seções
\renewcommand{\thesection}{\arabic{section}.}
\renewcommand{\thesubsection}{\thesection\arabic{subsection}}

%% hifenização BR
\usepackage[brazil]{babel}
\usepackage[T1]{fontenc}
%% Abaixo coloque como deveria ser
%% a hifenização
\hyphenation{Dev-Ops}

%% Força Imagem posição
\usepackage{float}
%% usar com \begin{figure}[H]

% Change these to change out the sort outputs.
\titulo{Aplicação de Assistentes Virtuais para Gerenciamento de Configuração}
\autor{Francismar Nascimento da Silva}
\orientador[Orientadora]{Profa. Dra. Fulana de Tal}
\coorientador{Prof. Dr. Beltrano da Silva}
\areaconcentracao{Engenharia de Software}
\data{09/2018}

% epigrafre, agradecimentos sao feitos 'na mao'.
% um dia eu faco alguns comandos para eles ;)

\begin{document} 

%\capa
\folhaderosto

%% sumario
%%\tableofcontents

%% aqui comeca o texto da monografia
%% voce pode dividir o conteudo em varios arquivos.
%% por exemplo, intro.tex, fundamentacao.tex, desenvolvimento.tex, conclusao.tex.
%% dai, vc inclui aqui assim: \input{intro.tex} e assim por diante.

\section{Introdução}
\begin{comment}A partir do desenvolvimento da ciência e das novas tecnologias, os meios de comunicação têm avançado significamente, modificando o formato da comunicação, neste contexto, a partir do século XX, a televisão e a internet proporcionaram a difusão dos conhecimentos e da comunicação no mundo.
\end{comment}
A forma como os usuários estão interagindo com as empresas evoluiu muito e rapidamente ao longo do tempo.
Durante anos, reuniões presenciais e telefonemas eram o meio de comunicação dominante. Então, com o surgimento da internet, uma infinidade de novas opções tornaram-se disponíveis: e-mails, aplicativos móveis, redes sociais e formulários para preenchimento online \cite{SalesforceDriftAudience2018TheReport}. 

Os aplicativos de mensagens tornaram o SMS um canal unidirecional utilizado, em sua maioria, pelas grandes companias para envio de notificações automáticas a seus clientes \cite{MobileTime2018MensageriaBrasil}
. Estes aplicativos, graças aos Chatbots, impulssionados por movimentos quase simultâneos de grandes empresas já estabelecidas, como Facebook, Google, Microsoft e Amazon, por um lado, e dos mais novos entrantes por outro: Slack, WeChat e Kikem chat \cite{MindBowser2017ChatbotSurvey}, são a próxima fronteira de interação entre usuários e empresas.

\begin{comment}
No Brasil, a Chatbos Brasil anunciou o resultado do primeiro Bots Brasil Awards, uma votação que escolheu os melhores Chatbots de 2017, os vencedores em suas categorias são: serviços (Magazine Luiza - Lú e Visa), e-Commerce (Pagseguro - Paquinho e ShopFácil), Mídia (Uol e Beta feminista), entretenimento (Rock in Rio - Roque) e Assistente Pessoal (Bia Talk e Clipping Bot).
\end{comment}

Mas, o que é um Chatbot? Chatbot é a junção das palavras: chat (conversa, bate-papo) e bot (robô). O Chatbot é uma entidade artificial projetada para simular uma conversa inteligente com parceiros humanos através da sua
linguagem natural \cite{Reshmi2016ImplementationBases}.

Os Chatbots, também chamados de Assistentes Virtuais (AV), podem ser utilizados para realizar desde tarefas simples, como agendamentos e reservas, ou mais complexas quando utilizados com a Inteligência Artificial.

Segundo \cite{Gartner2018GartnerYears}, embora o setor de atendimento ao cliente seja o que mais utiliza os Chatbots, outras áreas da empresa também podem ser beneficiadas. Quando os Chatbots são usados como interfaces de aplicativos, a maneira como trabalhamos mudará de ``o usuário tendo que aprender a interface'' para  ``o chatbot aprendendo o que o usuário quer''. 

\begin{comment}
Ainda sobre o futuro dos aplicativos de acordo com  \cite{Gartner2018GartnerGartner}, até 2021, mais de 50\% das empresas gastarão mais por ano na criação de Chatbots do que no desenvolvimento tradicional de aplicativos móveis, transformando a maneira como os aplicativos são construídos. 
\end{comment}
De acordo com \cite{GatewaySun2016Gartner:Robots},  em 2020, as pessoas não irão usar aplicativos em seus aparelhos. Na realidade, os apps estarão esquecidos. As pessoas vão contar com os assistentes virtuais pra tudo. Segundo a empresa a era pós-app está vindo.

As aplicações e benefícios de Assistentes Virtuais são inúmeros, desde suporte a clientes, coleta de dados, cuidados com a saúde, treinamentos, vendas, bancos e até ajudar as equipes de operação de TI de uma empresa, como veremos a seguir. 

Todas as empresas dependem da provisão de serviços de TI. Portanto, para manter a operação do negócio em funcionamento, é necessário gerenciar informações e relacionamentos que incluem uma infinidade de registros de itens de configuração (IC), IC é qualquer componente na infraestrutura de TI que precise ser configurado, geralmente são tarefas complexas e requerem o uso de ferramentas especialidadas. Estes desafios e a pressão do mercado fez com que as empresas buscassem soluções disruptivas para otimização máxima dos recursos e aumento da qualidade de entrega, tudo isso, fez surgir nas empresa uma nova cultura, conhecida como DevOps.
\begin{comment}
\begin{figure}[!htb]
\centering
\includegraphics [width=300px]{devops.jpg}
\caption{2013 IBM Corporation}
\end{figure}
\end{comment}

O Gerenciamento de Configuração (GC) surge como um recurso dentro  da abordagem DevOps, ou seja, as equipes de desenvolvimento (Dev) e operação (Ops) agora trabalham de forma colaborativa, utilizando práticas e recursos, dentre eles o GC, para entregar produtos e serviços. 

\begin{comment}Desta forma conseguem controlar e acompanhar mudanças, registrar a evolução dos projetos e manter a integração contínua.
\end{comment}

\begin{comment}O GC ajuda as empresas a manter a integridade e a consistência, além de permitir a escalabilidade e a entrega contínua, de serviços de hardware, software e rede, necessidades presentes em todos tipos de organizações, sejam elas do mundo dos negócios ou do mundo acadêmico. 
\end{comment}

Considerando a importância do GC no processo de funcionamento de toda empresa no contexto da Infraestrura de TI e os excelentes befícios que trazem quando utilizados com interfaces de aplicativos, propomos a implementação de uma aplicação de AVs como interface para o GC. Esta aplicação permitirá desde o monitoramento até a implementação de recursos nos Laboratórios de Informática e servidores da Universidade Federal de São Paulo-UNIFESP, campus São José dos Campos. 

Em virtude das informações até aqui apresentadas, este trabalho estabelece como problema de pesquisa responder: quais os tipos e aplicações de Assistentes Virtuais existentes,  destes quais são utilizados na solução de problemas que envolvem o Gerenciamento de Configuração e como é possível utilizar Assistentes Virtuais com o Gerenciamento de Configuração para o provisionamento de serviços de hardware, software e rede na Universidade Federal de São Paulo-UNIFESP, campus São José dos Campos?

\section{Objetivos}

\subsection{Geral}
Este trabalho visa identificar e analisar os tipos e aplicações existentes de Assistentes Virtuais que são utilizados para resolver problemas aderentes ao gerenciamento de configuração e implementar uma aplicação de Assistentes Virtuais para o Gerenciamento de configuração na Universidade Federal de São Paulo-UNIFESP, campus São José dos Campos.

\subsection{Específicos}
Para alcançar o seu objetivo central, este projeto está organizado em 4 capítulos sendo que no \textbf{Capítulo 1}, iremos conceituar Assistentes Virtuais, o seu surgimento e evolução, seus principais tipos, aplicações, casos de uso de sucesso e os benefícios que trazem para os usuários e empresas.
No \textbf{Capítulo 2}, iremos conceituar Gerenciamento de Configuração, seu surgimento e evolução, sua importância, os principais softwares que existem atualmente no mercado e como o GC pode ajudar as empresas a sobreviver num ambiente de alta competitividade e entrega contínua.
No \textbf{capítulo 3}, iremos relacionar e mostrar como os Assistentes Virtuais podem ser utilizados junto com o Gerenciamento de Configuração e agregar valor aos processos acadêmicos de tal forma que as equipes de desenvolvimento e operação possam trabalhar juntas e sincronizadas para provisionar serviços de hardware, software e rede mantendo a consistência e a integridade dos recursos. No \textbf{Capítulo 4}, vamos implementar uma solução de Assistentes virtuais para o Gerenciamento de configuração em uma aplicação que visar utilizar os benefícios dos assistentes virtuais junto com o Gerenciamento de configuração na Universidade Federal de São Paulo, Campus de São José dos Campos. E, por fim, no \textbf{Capítulo 5}, são apresentadas as considerações finais.

\section{Justificativa}

\begin{comment}Discutir a utilização de Assistentes Virtuais para o Gerenciamento de configuração, justifica-se pela necessidade de fornecer aos usuários interessados acesso não apenas a informações e recursos, que antes eram apenas de domínio das equipes de operação, mas também pela agilidade na entrega dos serviços.
\end{comment}
Os benefícios da utilização de Assistente Virtuais como interface de aplicativos são inúmeros dos quais podemos citar: disponibilidade 24 horas,  respostas instantâneas, respostas consistentes, respostas gravadas, transações instantâneas, paciência sem fim e programabilidade. 

\begin{figure}[H]
\centering
\includegraphics [width=350px]{chatbot-potential-benefits-1024x473}
\caption{The 2018 State of Chatbots Report: How Chatbots Are Reshaping Online Experiences}
\end{figure}

Como já dissemos anteriormente, o  GC é parte fundamental da cultura DevOps. 

A Figura 2 apresenta um estudo da empresa RightScale em 2016 acerca da adoção do  DevOps no mundo. O estudo já apontava números impressionantes – 84\% das grandes corporações e 72\% das pequenas e médias empresas já estavam adotando 
alguma prática DevOps, mostrando que o DevOps nos próximos anos estará presente em todas as empresas .

\begin{figure}[H]
\centering
\includegraphics [width=350px,height=200px]{devops-adoption}
\caption{RightScale 2016 nas Grandes e Pequenas e média empresas}
\end{figure}

Os serviços de virtualização, nuvem, containers, automação de servidores e  redes definidos por software são simplificados com a utilização do GC, facilitando o trabalho de operações de TI. Com o GC deve levar menos tempo e esforço  provisionar, configurar, atualizar e manter serviços. Os problemas devem ser detectados rapidamente e resolvidos. Os sistemas devem ser, consistentemente, configurados e atualizados. As equipes de TI devem gastar menos tempo em trabalho de rotina, ter tempo para fazer mudanças rapidamente e promover melhorias para ajudar suas organizações a atenderem às necessidades em constante mudança no mundo moderno.





A utilização de Assistentes Virtuais para o GC, faz com o que era antes restrito as equipes de operação, agora possa ser disponibilizado a todos os interessados.

\begin{comment}

Segundo ainda o mesmo relatório, para as empresas, os benefícios também são inegáveis, a saber:

1 - \textbf{Maior satisfação do cliente}: todos os benefícios acima resultarão em maior satisfação do cliente, o que pode levar a uma maior defesa e venda ao cliente.

2- \textbf{Redução de custos}:  A necessidade das empresas de desenvolver o departamento de atendimento ao cliente pode ser gerenciada com a implantação de bots cada vez mais capazes, que lidam com consultas cada vez mais complexas.

3- \textbf{Maior interação com o cliente e vendas}: os  Bots fornecem outro canal para alcançar seus clientes. Os bots podem ser aproveitados para aumentar o envolvimento do cliente com dicas e ofertas oportunas.

4- \textbf{Alcançando novos clientes}:  Plataformas bot como Kik ou Facebook Messenger são um dos aplicativos mais populares. Estar continuamente ativo nessas plataformas ajuda as empresas a alcançar novos clientes que, de outra forma, não desejam entrar em contato com a empresa com um e-mail ou uma ligação.

5- \textbf{Obter uma compreensão mais profunda dos clientes}:  seus clientes raramente conversam com sua empresa. Os chatbots fornecem à sua empresa registros detalhados e acionáveis dos maiores pontos problemáticos de seus clientes, ajudando sua empresa a melhorar seus produtos e serviços.
\end{comment}


Atualmente, na UNIFESP campus São José dos Campos, o processo completo de clonagem de uma máquina de laboratório demora 4 horas, consumindo tempo e degradando o desempenho da rede de computadores. Com o uso do GC os recursos seriam implementados conforme a demanda, em dias e horários adequados, e até mesmo monitorados conforme a necessidade. 

Um Assistente Virtual poderia dispor aos usuários a instalação desde um simples pacote de bibliotecas, a um ambiente completo de desenvolvimento com banco de dados, servidor web, aplicação para o desenvolvimento, tudo isso ainda utilizando conteineres.

Pesquisadores, docentes, discentes e a propria instituição como um todo são beneficiados com informações e serviços a medida de suas necessidades, com base nas informações coletadas e no interesse dos usuários é possível definir a melhor forma de utilização desses recursos. 

No ambiente acadêmico o acesso dos docentes, discentes e interessados, a dados e recursos, podem ser facilitados com a utilização dos Assistentes Virtuais. Ajudando-os na organização, provisionamento e na preparação das aulas.

Assim, o presente trabalho parte da necessidade de entender os diferentes aspectos que norteiam esses 2 mundos afim de integrá-los numa solução que fornecesse o melhor de cada um. Os interessados podem se beneficiar e obter respostas a problemas ligados as suas área de interesse.

\begin{comment}

De acordo com \cite{Gartner2018GartnerYears}
, os Chatbots deverão apresentar um enorme crescimento nos próximos anos. Embora menos de 4\% das organizações já tenham implantado interfaces conversacionais (incluindo chatbots), 38\% das organizações estão planejando implementar ou experimentando ativamente a tecnologia. 

Na prática  AV podem diminuir o tempo de espera, minimizar erros, dar agilidade e aumentar a satisfação dos usuários.

O GC é parte importante da cultura DevOps que visa juntar, em colaboração contínua, as equipes de desenvolvimento (Dev) e operação (Ops). (explicar melhor)Segundo  previu que a partir de 2016 o DevOps se estabeleceria como uma disciplina dominante e seria adotado por 25\% das organizações do Mundo. A Figura 1.2 apresenta um estudo da empresa RightScale disponibilizada anualmente que tem como principal objetivo analisar como está a adoção de Cloud e DevOps no mundo. No ano passado, chegamos a números impressionantes – 81\% das grandes corporações e 70\% das pequenas e médias empresas estão adotando DevOps -, que nos mostram que nos próximos anos é possível que o DevOps esteja presente em todas as empresas que buscam acompanhar a velocidade e a competitividade do mercado atual.
\begin{figure}[!htb]
\centering
\includegraphics [width=450px]{devops-adoption}
\caption{RightScale 2016 nas Grandes e Pequenas e média empresas}
\end{figure}

O GC transforma a infraestruta em código (Infracture as Code - IAC), isto significa que vamos escrever linhas de código para automatizar o provisionamento da infraestrutura e das implantações. 

\end{comment}

\section{Metodologia}
O presente trabalho consiste em uma pesquisa aplicada de caráter exploratório e descritivo com base em um estudo comparativo do conteúdo das obras de diferentes autores em uma revisão ( sistemática da literatura / bibliográfica ) que
permita um maior aprofundamento sobre o tema da pesquisa. Sem a pretensão de
estabelecer um discurso conclusivo sobre as questões pesquisadas, busca-se analisar
os conceitos chave tratados nesta dissertação, contribuindo com novas reflexões e 
perspectivas de estudo.

\section{Recursos necessários}
Os recursos necessários para o desenvolvimento, análises  e aplicação deste projeto estão disponíveis na infraestrutura da Universidade Federal de São Paulo-UNIFESP, campus São José dos Campos, onde pretende-se aplicar este projeto, tais como:
\begin{itemize}
\item Laboratórios de informática 
\item Servidores
\end{itemize}

%%-------------------------------------------------
%% Referências
%% coloque aqui o seu arquivo .bib
%% IMPORTANTE: nao use bibliographystyle!
%% o estilo ja vem definido.
\bibliography{references}

%% de acordo com o template, o glossario vem
%% depois das referencias e deve estar em ordem
%% alfabetica.
%% depois de muito esforco consegui fazer com que
%% o glossario ficasse em ordem alfabetica automaticamente.
%% ainda nao sei a escalabilidade do algoritmo :(

%% DICA: voce pode ir definindo os acronimos ao longo do texto.
%% Por exemplo, no capitulo 1, vc ta escrevendo:
%% Segundo Fulano, Model-Driven Development (MDD)\acronym{MDD}{Model-Driven Development} � uma t�cnica bla bla bla...

\end{document}